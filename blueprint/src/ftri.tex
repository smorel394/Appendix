\chapter{Section A.1: filtered triangulated categories}

In Appendix~A of~\cite{Be1}, Beilinson introduced filtered triangulated 
categories over triangulated categories, that have all the abstract properties of filtered
derived categories, and generalized the properties of filtered derived categories to this more general setting. We review his definition and results.
Note that Section~6 of Schn\"urer's paper \cite{Schnur} gives more detailed proofs of many of the results of Appendix~A of~\cite{Be1}.

This section corresponds to the Lean file \url{Filtered_no_proof.lean}.

\section{Definition A.1.1}

When formalizing the definition of filtered triangulated 
categories, we run into our first issue immediately. Beilinson defines an filtered triangulated category as a triangulated
category $\DF$ with a second shift functor $s$, which should be a triangulated self-equivalence (plus some extra structure). In particular,
the functor $s$ needs to commute with the shifts coming from the triangulated category structure. The easiest way to encode all
the necessary compatibilities is actually to put a shift by $\Z \times \Z$ on our category $\DF$, where the shift by the first factor
will be part of the triangulated structure and the shift by the second factor will give the functor $s$. The beginning of the project
is devoted to setting this shift structure. In particular, we chose to make the default shift by $\Z$ on the  category $\DF$ to be the
shift by the first factor, and to introduce a type synonym `FilteredShift $\DF$' which carries a shift by $\Z$ encoding the functor  $s$.

The following is part of Definition~A.1 of~\cite{Be1}. (Almost: we actually define filtered pretriangulated categories in the Lean code,
and a filtered triangulated category is a filtered pretriangulated category that is also triangulated.)

\begin{definition}
% Definition A.1.1(1).
\label{def-FilteredTriangulated}
\lean{CategoryTheory.FilteredTriangulated}
\leanok
A \textit{filtered} \textit{triangulated} \textit{category} is the data of:
\begin{itemize}
\item a triangulated category $\DF$;
\item two full triangulated subcategories $\DF(\leq 0)$, $\DF(\geq 0)$ of $\DF$ that are stable by isomorphisms;
\item a triangulated self-equivalence $s\colon \DF\ra\DF$ (called \emph{shift of filtration});
\item a morphism of functors $\alpha\colon \id_{\DF}\ra s$;

\end{itemize}
satisfying the following conditions, where, for every $n\in\Z$, we set
\[\DF(\leq n)=s^n\DF(\leq 0)\mbox{ and }\DF(\geq n)=s^n\DF(\geq 0):\]
\begin{itemize}
\item[(i)] We have $\DF(\geq 1)\subset\DF(\geq 0)$, $\DF(\leq 1)\supset\DF(\leq 0)$ and
\[\DF=\bigcup_{n\in\Z}\DF(\leq n)=\bigcup_{n\in\Z}\DF(\geq n).\]
\item[(ii)] For any $X\in\Ob\DF$, we have $\alpha_X=s(\alpha_{s^{-1}(X)})$.
\item[(iii)] For any $X\in\Ob\DF(\geq 1)$ and $Y\in\Ob\DF(\leq 0)$, we have $\Hom(X,Y)=0$, and the maps $\Hom(s(Y),X)\ra\Hom(Y,X)\ra\Hom(Y,s^{-1}(X))$ induced
by $\alpha_Y$ and $\alpha_{s^{-1}(X)}$ are bijective.
\item[(iv)] For every $X\in\Ob\DF$, there exists a distinguished triangle $A\ra X\ra B\xra{+1}$ with $A$ in $\DF(\geq 1)$ and $B$ in $\DF(\leq 0)$.
\end{itemize}

\end{definition}

We need to setup extra definitions (for example, classes like `IsLE' and `IsGE'), then prove a lot of "easy" lemmas in order
to make this definition usable. For example, there is a lemma \url{CategoryTheory.FilteredTriangulated.exists_triangle} saying that, for
every $X$ and $\DF$ and every $n\in\Z$, there exists a distinguished triangle $A\ra X\ra B \ra A[1]$ with $A$ in $\DF(\ge n + 1)$
and $B$ in $\DF(\le n)$.

Then we define filtered triangulated functors. In Lean, we first need to ask for a `CommShift' structure by $\Z\times\Z$, and then 
we require three properties: preservation of objects that are $\le 0$, preservation of objects that are $\ge 0$, and commutation with
the natural transformation $\alpha$.

\begin{definition}
% Definition A.1.1(2).
\label{def-FilteredTriangulatedFunctor}
\lean{CategoryTheory.Functor.IsFilteredTriangulated}
\leanok
\uses{{def-FilteredTriangulated}}
If $\DF$ and $\DF'$ are filtered triangulated categories, a \emph{filtered triangulated functor} from $\DF$ to $\DF'$ is the data of a triangulated functor $T\colon \DF\ra\DF'$ and a natural isomorphism
$s'\circ T\iso T\circ s$ 
such that $T(\DF(\leq 0))\subset\DF'(\leq 0)$,
$T(\DF(\geq 0))\subset\DF(\geq 0)$ and that, for every $X\in\Ob\DF$, the following triangle commutes:

\begin{tikzcd}
T(X) \arrow[r, "\alpha'_{T(X)}"] \arrow[rd, "T(\alpha(X))"'] 
& s'(T(X))\arrow[d, "\wr"] \\
& T(s(X))
\end{tikzcd}

\end{definition}


\begin{definition}
% Definition A.1.1(3) part 1.
\label{def-FilteredTriangulatedOver}
\lean{CategoryTheory.isFilteredTriangulated_over}
\leanok
\uses{{def-FilteredTriangulated}}
Let $\Df$ be a triangulated category. A \emph{filtered triangulated category over $\Df$} is a filtered triangulated category $\DF$ together 
with a fully faithful functor $i\colon \Df\ra\DF$ whose essential image is $\DF(\leq 0)\cap\DF(\geq 0)$.

\end{definition}

\begin{definition}
\label{def-FilteredTriangulatedOver-equivalence}
\lean{CategoryTheory.isFilteredTriangulated_over_equiv}
\leanok
\uses{def-FilteredTriangulatedOver}
Let $\Df$ be triangulated categories, $\DF$ be a filtered triangulated category over $\Df$. We write $equiv$ for the equivalence of categories 
$\Df\ra \DF(\leq 0)\cap\DF(\geq 0)$ induced by the fully faithful functor $i\colon \Df\ra\DF$.

\end{definition}

\begin{definition}
% Definition A.1.1(3) part 2.
\label{def-FilteredLifting}
\lean{CategoryTheory.Functor.filteredLifting}
\leanok 
\uses{def-FilteredTriangulatedOver, def-FilteredTriangulatedFunctor}
Let $\Df, \Df'$ be triangulated categories, $\DF$ (resp. $\DF'$) be a filtered triangulated category over $\Df$ (resp. $\Df'$) 
and $T\colon \Df\ra\Df'$ be a triangulated functor. A \emph{filtered lifting} of $T$ is a filtered triangulated functor
$FT\colon \DF\ra\DF'$ and a natural isomorphism $compat : i'\circ T\simeq TF\circ i$.

\end{definition}

I am guessing that the isomorphism $compat$ should satisfy some more compatibilities,
notably with the "commutation with shifts" isomorphisms. (This will probably come up when I actually formalize
properties of filtered triangulated functors.)


\section{Proposition A.1.3}

We fix a filtered triangulated catgeory $\DF$.

\begin{proposition}
% Proposition A.1.3(i) first sentence.
\label{prop-A.1.3.1-le}
\lean{CategoryTheory.FilteredTriangulated.LE_reflective}
\leanok
\uses{def-FilteredTriangulated}
For every $n\in\Z$, the full subcategory $\DF(\le n)$ is reflective.

\end{proposition}

\begin{proposition}
% Proposition A.1.3(i) first sentence.
\label{prop-A.1.3.1-ge}
\lean{CategoryTheory.FilteredTriangulated.GE_coreflective}
\leanok
\uses{def-FilteredTriangulated}

For every $n\in\Z$, the full subcategory $\DF(\le n)$ is coreflective.

\end{proposition}

\begin{definition}
% Proposition A.1.3.1.
\label{def-truncLE}
\lean{CategoryTheory.FilteredTriangulated.truncLE}
\leanok
\uses{prop-A.1.3.1-le}

For every $n\in\Z$, we denote by $\sigma_{\le n} : \DF \ra \DF$ the composition of a left adjoint of the inclusion functor $\iota:\DF(\le n)\to\DF$
and of $\iota$.

\end{definition}

\begin{definition}
% Proposition A.1.3.1.
\label{def-truncLEπ}
\lean{CategoryTheory.FilteredTriangulated.truncLEπ}
\leanok
\uses{{def-truncLE}}

For every $n\in\Z$, we denote by $truncLEπ\ n : {\ungras}_{\DF} \ra \sigma_{\le n}$ the unit of the adjunction.

\end{definition}

\begin{definition}
% Proposition A.1.3.1.
\label{def-truncGE}
\lean{CategoryTheory.FilteredTriangulated.truncGE}
\leanok
\uses{prop-A.1.3.1-ge}

For every $n\in\Z$, we denote by $\sigma_{\ge n} : \DF \ra \DF$ the composition of a right adjoint of the inclusion functor $\iota:\DF(\ge n)\to\DF$
and of $\iota$.

\end{definition}

\begin{definition}
% Proposition A.1.3.1.
\label{def-truncGEι}
\lean{CategoryTheory.FilteredTriangulated.truncGEι}
\leanok
\uses{def-truncGE}


\end{definition}

We have some lemmas about these, among others concerning the essential image of $\sigma_{\le n}$ and $\sigma_{\ge n}$, as well as lemmas
expressing the universal property of the adjunctions.

\begin{definition}
% Proposition A.1.3.1, second sentence.
\label{def-truncLE_shift}
\lean{CategoryTheory.FilteredTriangulated.instCommShiftTruncLEInt}
\leanok
\uses{def-truncLE}

For every $n\in\Z$, the functor $\sigma_{\le n} : \DF \ra \DF$ commutes with the triangulated shift.

\end{definition}

\begin{proposition}
% Proposition A.1.3.1, second sentence.
\label{prop-truncLE_tri}
\lean{CategoryTheory.FilteredTriangulated.instIsTriangulatedTruncLE}
\leanok
\uses{def-truncLE_shift}

For every $n\in\Z$, the functor $\sigma_{\le n} : \DF \ra \DF$ is triangulated.

\end{proposition}

\begin{definition}
% Proposition A.1.3.1, second sentence.
\label{def-truncGE_shift}
\lean{CategoryTheory.FilteredTriangulated.instCommShiftTruncGEInt}
\leanok
\uses{def-truncGE}

For every $n\in\Z$, the functor $\sigma_{\ge n} : \DF \ra \DF$ commutes with the triangulated shift.

\end{definition}

\begin{proposition}
% Proposition A.1.3.1, second sentence.
\label{prop-truncGE_tri}
\lean{CategoryTheory.FilteredTriangulated.instIsTriangulatedTruncGE}
\leanok
\uses{def-truncGE_shift}

For every $n\in\Z$, the functor $\sigma_{\ge n} : \DF \ra \DF$ is triangulated.

\end{proposition}

\begin{proposition}
% Proposition A.1.3.1, second sentence.
\label{prop-truncLE_LE}
\lean{CategoryTheory.FilteredTriangulated.instIsLEObjTruncLE_1}
\leanok
\uses{def-truncLE}

For all $n,m\in\Z$, the functor $\sigma_{\le n} : \DF \ra \DF$ sends $\DF(\le m)$ to itself.

\end{proposition}

\begin{proposition}
% Proposition A.1.3.1, second sentence.
\label{prop-truncLE_GE}
\lean{CategoryTheory.FilteredTriangulated.instIsGEObjTruncLE}
\leanok
\uses{def-truncLE}

For all $n,m\in\Z$, the functor $\sigma_{\le n} : \DF \ra \DF$ sends $\DF(\ge m)$ to itself.

\end{proposition}

\begin{proposition}
% Proposition A.1.3.1, second sentence.
\label{prop-truncGE_LE}
\lean{CategoryTheory.FilteredTriangulated.instIsLEObjTruncGE}
\leanok
\uses{def-truncGE}

For all $n,m\in\Z$, the functor $\sigma_{\ge n} : \DF \ra \DF$ sends $\DF(\le m)$ to itself.

\end{proposition}

\begin{proposition}
% Proposition A.1.3.1, second sentence.
\label{prop-truncGE_GE}
\lean{CategoryTheory.FilteredTriangulated.instIsGEObjTruncGE_1}
\leanok
\uses{def-truncGE}

For all $n,m\in\Z$, the functor $\sigma_{\ge n} : \DF \ra \DF$ sends $\DF(\ge m)$ to itself.

\end{proposition}

We need to switch the order of statements compared to the paper, because the proof of Proposition A.1.3 (ii) uses Proposition A.1.3 (iii),
but with general indices.

The way we state the existence of the triangle cheating in a way, because the connecting morphism in the triangle is not arbitrary,
it's given by the axioms. (The statements are still okay thanks to the uniqueness.)


\begin{definition}
% Proposition A.1.3.3, second sentence.
\label{def-triangleGELE_delta}
\lean{CategoryTheory.FilteredTriangulated.truncLEδGE}
\leanok
\uses{def-truncLE, def-truncGE}

For $n\in\Z$, this is the natural transformation $\delta\colon \sigma_{\le n} \to 
(shiftFunctor\ 1) \circ (\sigma_{\ge n + 1})$.

\end{definition}

\begin{definition}
% Proposition A.1.3.3, second sentence.
\label{def-triangleGELE}
\lean{CategoryTheory.FilteredTriangulated.triangleGELE}
\leanok
\uses{def-triangleGELE_delta}

For $n\in\Z$, this is the functor $triangleGELE n$ from $\DF$ to the category of triangles of $\DF$ sending $X$ to the triangle 
\[\sigma_{\ge n + 1} X \xra{truncGEι} X \xra{truncLEπ} \sigma_{\le n} \xra{\delta} (\sigma_{\ge n + 1} X)[1].\]

\end{definition}

\begin{proposition}
% Proposition A.1.3.3, second sentence.
\label{def-triangleGELE_dist}
\lean{CategoryTheory.FilteredTriangulated.triangleGELE_distinguished}
\leanok
\uses{def-triangleGELE}

For $X$ in $\DF$ and $n\in\Z$, the triangle $triangleGELE n X$ is distinguished.

\end{proposition}

The second part of Proposition A.1.3(iii) is a uniqueness statement for the triangle.
In the paper, this says that any distinguished triangle $A \ra X \ra B \ra A[1]$ with $A$ in $\DF(≤ n)$ 
and $B$ in $\DF(≥ n + 1)$ is isomorphic to $triangleGELE n X$ in a unique way. Actually, this is not
quite correct, because we only have uniqueness if we require the morphism of triangles
to be the identity of $X$ on the second objects. Also, the other morphisms are already explicit and
uniquely determined, they are given by $descTruncLE$ and $liftTruncGE$, so the real content
is that these morphisms are isomorphisms.

\begin{proposition}
% Proposition A.1.3.3, second sentence.
\label{prop-triangleGELE_uniq_left}
\lean{CategoryTheory.FilteredTriangulated.isIso_descTruncLE_of_fiber_ge}
\leanok
\uses{def-truncLE, def-truncGE}

Let $n\in\Z$ and $T$ be a distinguished triangle in $\DF$. Assume that the first object $T.X_1$ of $T$ is in $\DF(\ge n+1)$
and that the third object $T.X_3$ of $T$ is in $\DF(\le n)$. Then the morphism $\sigma_{\le n} T.X_2 \ra T.X_3$ induced by the second 
morphism of $T$ and the universal property of $truncLE$ is an isomorphism.

\end{proposition}

\begin{proposition}
% Proposition A.1.3.3, second sentence.
\label{prop-triangleGELE_uniq_right}
\lean{CategoryTheory.FilteredTriangulated.isIso_liftTruncGE_of_cone_le}
\leanok
\uses{def-truncLE, def-truncGE}

Let $n\in\Z$ and $T$ be a distinguished triangle in $\DF$. Assume that the first object $T.X_1$ of $T$ is in $\DF(\ge n+1)$
and that the third object $T.X_3$ of $T$ is in $\DF(\le n)$. Then the morphism $T.X_1\ra \sigma_{\ge n + 1} T.X_2$ induced by the first 
morphism of $T$ and the universal property of $truncGE$ is an isomorphism.

\end{proposition}

Proposition A.1.3(ii) asserts that the functors $\sigma_{\le n}$ and $\sigma_{\ge m}$ commute.
Before proving it, we establish a criterion for triangulated endofunctors of $\DF$
to commute with the truncation functors (up to an isomorphism which will arise naturally).
It is better to make this more general, as it will be used again.

We consider another filtered triangulated category $\DF'$.

\begin{definition}
% Prepration to Proposition A.1.3.2.
\label{def-commute_truncLE}
\lean{CategoryTheory.FilteredTriangulated.commute_truncLE}
\leanok
\uses{def-truncLE, def-FilteredTriangulatedFunctor}

Let $F:\DF\ra\DF'$ be a filtered triangulated functor
and $n\in\Z$. If $F$ preserves the subcategory $\DF(\le n)$, we 
get a natural transformation
$\sigma_{\le n} \circ F \ra F \circ\sigma_{\le n}$.

\end{definition}


\begin{definition}
% Prepration to Proposition A.1.3.2.
\label{def-commute_truncGE}
\lean{CategoryTheory.FilteredTriangulated.commute_truncGE}
\leanok
\uses{def-truncGE, def-FilteredTriangulatedFunctor}

Let $F:\DF\ra\DF'$ be a filtered triangulated functor
and $n\in\Z$. If $F$ preserves the subcategory $\DF(\ge n)$, we 
get a natural transformation
$F \circ\sigma_{\ge n} \ra \sigma_{\ge n} \circ F$.

\end{definition}

\begin{remark}
It would be natural to use mates in this construction, but the properties were easier to prove with the direct definition
we adopted.
\end{remark}


\begin{proposition}
% Proposition A.1.3.2.
\label{prop-commute_truncLE_iso}
\lean{CategoryTheory.FilteredTriangulated.isIso_commute_truncLE}
\leanok
\uses{def-commute_truncLE}

Let $F:\DF\ra\DF'$ be a filtered triangulated functor
and $n\in\Z$. Suppose that $F$ preserves the subcategories $\DF(\le n)$.
and $\DF(\ge n + 1)$.

Then the natural transformation
$\sigma_{\le n} \circ F \ra F \circ\sigma_{\le n}$ is an isomorphism.

\end{proposition}


\begin{proposition}
% Proposition A.1.3.2.
\label{prop-commute_truncGE_iso}
\lean{CategoryTheory.FilteredTriangulated.isIso_commute_truncGE}
\leanok
\uses{def-commute_truncGE}

Let $F:\DF\ra\DF'$ be a filtered triangulated functor
and $n\in\Z$. Suppose that preserves the subcategories $\DF(\le n)$.
and $\DF(\ge n + 1)$.

Then the natural transformation
$F \circ\sigma_{\ge n} \ra \sigma_{\ge n} \circ F$ is an isomorphism.

\end{proposition}

Now we write the existence statement of Proposition A.1.3(ii).

\begin{definition}
% Definition in Proposition A.1.3.2.
\label{def-truncLEGE}
\lean{CategoryTheory.FilteredTriangulated.truncLEGE}
\leanok
\uses{def-truncLE, def-truncGE}

For $a,b\in\Z$, we define $\sigma'_{[a,b]} : \DF\ra\DF$ as $\sigma_{\leq b}\circ\sigma_{\geq a}$.

\end{definition}

\begin{definition}
% Definition in Proposition A.1.3.2.
\label{def-truncGELE}
\lean{CategoryTheory.FilteredTriangulated.truncGELE}
\leanok
\uses{def-truncLE, def-truncGE}

For $a,b\in\Z$, we define $\sigma_{[a,b]} : \DF\ra\DF$ as $\sigma_{\geq a}\circ\sigma_{\leq b}$.

\end{definition}

\begin{definition}
% Definition in Proposition A.1.3.2.
\label{def-LEGEToGELE}
\lean{CategoryTheory.FilteredTriangulated.truncLEGEToGELE}
\leanok
\uses{def-truncLEGE, def-truncGELE, def-commute_truncLE}

For $a,b\in\Z$, we have a natural transformation $\sigma'_{[a,b]} \ra\sigma_{[a,b]}$ given by \url{commute_truncLE}.

\end{definition}

\begin{definition}
% Proposition A.1.3.2.
\label{def-LEGEToGELE_iso}
\lean{CategoryTheory.FilteredTriangulated.truncLEGEIsoGELE}
\leanok
\uses{def-LEGEToGELE}

For $a,b\in\Z$, the natural transformation $\sigma'_{[a,b]} \ra\sigma_{[a,b]}$ of Definition \ref{def-LEGEToGELE} is an isomorphism.

\end{definition}

\begin{remark}
Because $\sigma_{[a,b]}$ and $\sigma'_{[a,b]}$ are isomorphic (by a "canonical" isomorphism), they only get 
one notation in the paper, but of course you need two notations in Lean.

\end{remark}

The uniqueness statement of Proposition A.1.3(ii):

\begin{proposition}
% Proposition A.1.3.2.
\label{prop-LEGEToGELE_uniq}
\lean{CategoryTheory.FilteredTriangulated.truncLEGEToGELE_uniq}
\leanok
\uses{def-LEGEToGELE}

Let $a,b\in\Z$, let $X$ be an object of $\DF$, and let $f:\sigma'_{[a,b]} X\ra\sigma_{[a,b]}X$. Suppose that
$f$ makes the following diagram commute: 

\begin{tikzcd}
\sigma_{\geq b}\ar[rr]\ar[rd] & & \id_{\DF}\ar[rr] & & \sigma_{\leq a} \\
& \sigma_{\leq a}\sigma_{\geq b}\ar[rr, "f"'] && \sigma_{\geq b}\sigma_{\leq a}\ar[ru] &
\end{tikzcd}

Then $f$ is equal to the morphism of Definition~\ref{def-LEGEToGELE}.

\end{proposition}

We can now state a more general version of Proposition A.1.3(iii).

\begin{definition}
\label{def-truncGELE_le_up}
\lean{CategoryTheory.FilteredTriangulated.truncGELE_le_up}
\leanok
\uses{def-truncGELE}

Let $a,b,c\in\Z$ with $b\le c$. Then we have a natural transformation $\sigma_{[a,c]} \ra \sigma_{[a,b]}$.

\end{definition}

\begin{definition}
\label{def-truncGELE_le_down}
\lean{CategoryTheory.FilteredTriangulated.truncGELE_le_down}
\leanok
\uses{def-truncGELE}

Let $a,b,c\in\Z$ with $a\le b$. Then we have a natural transformation $\sigma_{[b,c]} \ra \sigma_{[a,c]}$.

\end{definition}

\begin{definition}
\label{def-truncGELE_δ}
\lean{CategoryTheory.FilteredTriangulated.truncGELE_δ}
\leanok
\uses{def-truncGELE}

For $a,b,c\in\Z$, we have a natural transformation
$\delta\colon \sigma_{[a,b]} \to (shiftFunctor\ 1)\circ(\sigma_{[b+1,c]})$.

\end{definition}

\begin{definition}
\label{def-truncGELE_triangle}
\lean{CategoryTheory.FilteredTriangulated.truncGELE_triangle}
\leanok
\uses{def-truncGELE_δ, def-truncGELE_le_up, def-truncGELE_le_down}

For $a,b,c\in\Z$ such that $a\le b\le c$,
this is the functor $triangleGELE a b c$ from $\DF$ to the category of triangles of $\DF$ sending $X$ to the triangle 
\[\sigma_{[b+1,c]} X \ra \sigma_{[a,c]} X \ra \sigma_{[a,b]} \xra{\delta} (\sigma_{[b+1,c]} X)[1].\]

\end{definition}

\begin{proposition}
\label{def-truncGELE_triangle_dist}
\lean{CategoryTheory.FilteredTriangulated.truncGELE_triangle_distinguished}
\leanok
\uses{def-truncGELE_triangle}
For $X$ in $\DF$ and $a,b,c\in\Z$ such that $a\le b\le c$, the triangle of Definition \ref{def-truncGELE_triangle}
is distinguished.

\end{proposition}

We did not write the uniqueness statement for the triangle, though there is a statement similar to 
Propositions~\ref{prop-triangleGELE_uniq_left} and ~\ref{prop-triangleGELE_uniq_right}.

Proposition A.1.3 (iv) uses the adjective "canonical" and an equality sign, which is bad. We need to explain what 
compatibilities hide under it, and to make the equality sign by an isomorphism 
(that will be given by the universal property of the adjoint).

Also, we actually want the isomorphisms for "second" shifts
by any integer, compatible with the zero and the addition, like in `Functor.CommShift`.
We introduce a new structure for this, similar to \url{Functor.CommShift} and called
\url{familyCommShift}. It expresses the fact that a family of endofunctors $F: \Z \ra\DF$ has a family of 
isomorphisms $F (n + m) \circ shiftFunctor_2 m \iso shiftFunctor_2 m \circ F n$, where $shiftFunctor_2 a$ is the
second shift by $a\in\Z$ (i.e., $s^a$), and that these isomorphisms are equal to the obvious ones when $m=0$ and 
compatible with addition.

\begin{definition}
% Proposition A.1.3.4.
\label{def-truncLE_commShift}
\lean{CategoryTheory.FilteredTriangulated.truncLE_commShift}
\leanok
\uses{def-truncLE}

The family of functors $n \mapsto truncLE n$ has a \url{familyCommShift} structure.

\end{definition}

\begin{definition}
% Proposition A.1.3.4.
\label{def-truncGE_commShift}
\lean{CategoryTheory.FilteredTriangulated.truncGE_commShift}
\leanok
\uses{def-truncGE}

The family of functors $n \mapsto truncGE n$ has a \url{familyCommShift} structure.

\end{definition}


\section{The "forget the filtration" functor (Proposition A.1.6)}

The next thing in the paper is the definition, when we have a filtered triangulated category
$\DF$ over a triangulated category $\Df$, of the "graded pieces" functors $\Gra^n : \DF \to \Df$, 
which use an arbitrary quasi-inverse of the fully faithful functor $i:\Df\to\DF$ on the essential image of
$i$.

Rather than using an arbitrary quasi-inverse, it makes things much simpler to use the one
given by the "forget the filtration" functor $\omega : \DF \to \Df$, which has the
additional pleasant property that it is defined on all of $\DF$ and so avoids an
\url{ObjectProperty.lift}. In fact, this is even better, as $\omega$ intertwines the
second shift and the identity of $\Df$, so we can directy define $\Gra^n$ as $\omega\circ\sigma_{[n,n]}$
(noting that $\Gra^n$ was originally defined as $shiftFunctor_2 (-n)\circ\sigma_{[n,n]}$ followed 
by a quasi-inverse of the equivalence $\Df\iso \DF(\le 0)\cap\DF(\ge 0)$).

For this, we need to change the order of statements and do Proposition A.1.6 first (this is
possible as that proposition makes no use of the functors $\Gra^n$).

%TODO: this part needs to be rewritten. First we should define $\omega$ on $\DF(\le 0)$ as a left adjoint, then
%extend it to $\DF$ using compatibility with $s$, then state that it is a right adjoint on $\DF(\ge 0)$.

\begin{definition}
\label{def-FF}
\lean{CategoryTheory.FilteredTriangulated.ForgetFiltration}
\uses{def-FilteredTriangulatedOver}

The functor $\omega:\DF\to\Df$.

\end{definition}

\begin{definition}
%Proposition A.1.6(a)
\label{def-FF-leftAdjoint}
\lean{CategoryTheory.FilteredTriangulated.ForgetFiltration_leftAdjoint}
\uses{def-FF}

The functor $\omega:\DF\to\Df$ restricted to the full subcategory $\DF(\le 0)$ is left adjoint to 
the functor $\Df \to \DF(\le 0)\cap\DF(\ge 0)\subset\DF(\le 0)$.

\end{definition}

\begin{definition}
%Proposition A.1.6(b)
\label{def-FF-rightAdjoint}
\lean{CategoryTheory.FilteredTriangulated.ForgetFiltration_rightAdjoint}
\uses{def-FF}

The functor $\omega:\DF\to\Df$ restricted to the full subcategory $\DF(\ge 0)$ is right adjoint to 
the functor $\Df \to \DF(\le 0)\cap\DF(\ge 0)\subset\DF(\ge 0)$.

\end{definition}

\begin{proposition}
%Proposition A.1.6(c)
\label{prop-FF-shift}
\lean{CategoryTheory.FilteredTriangulated.ForgetFiltration_shift}
\uses{def-FF}

For every $X$ in $\DF$, the image by $\omega$ of the map $\alpha(X) : X \to s(X)$ is an
isomorphism. 

\end{proposition}

This implies that $\omega$ intertwines the second shift $s$ and the identity of $\Df$.
Right now this is expressed via a custom structure called a \url{leftCommShift}, but I 
don't know if that's optimal.


\begin{proposition}
%Proposition A.1.6(d)
\label{prop-FF-ff}
\lean{CategoryTheory.FilteredTriangulated.ForgetFiltration_ff}
\uses{def-FF}

Let $X,Y$ be objects of $\DF$. If $X$ is in $\DF(\le 0)$ and $Y$ is in $\DF(\ge 0)$,
then the map $\Hom_{\DF}(X,Y) \to\Hom_{\Df}(\omega(X),\omega(Y))$ is bijective.

\end{proposition}

The functor $\omega$ should also be triangulated.
(This actually follows from the other conditions, but is
not stated in the paper. Note that the first statement contains
data, so I am actually cheating here, because the data is determined
by the other properties of $\omega$.)

\begin{definition}
\label{def-FF-commShift}
\lean{CategoryTheory.FilteredTriangulated.instCommShiftForgetFiltrationInt}
\uses{def-FF}

The functor $\omega:\DF\to\Df$ commutes with the triangulated shifts.

\end{definition}

\begin{proposition}
\label{prop-FF-tri}
\lean{CategoryTheory.FilteredTriangulated.instIsTriangulatedForgetFiltration}
\uses{def-FF-commShift}

The functor $\omega:\DF\to\Df$ is triangulated.

\end{proposition}

We don't write the uniqueness statements here, as they are painful (which probably means
that I haven't yet found the correct way to talk about $\omega$).

Property (a) implies that we have an isomorphism $\omega\circ i \iso {\ungras}_{\Df}$ given by the
counit of the adjunction of Definition~\ref{def-FF-leftAdjoint}.
Property (b) gives an isomorphism in the other direction, given by the unit of the adjunction of
Definition~\ref{def-FF-rightAdjoint}.
We give the definition of the first
isomorphism,
and the result after that
says that these isomorphisms are inverses of each other. This compatibility is not stated 
in the paper.

\begin{definition}
\label{def-FF-i}
\lean{CategoryTheory.FilteredTriangulated.ForgetFiltration_functor}
\uses{def-FF-leftAdjoint}

The isomorphism $\omega\circ i \iso {\ungras}_{\Df}$ given by the
counit of the adjunction of Definition~\ref{def-FF-leftAdjoint}.

\end{definition}

\begin{proposition}
\label{prop-FF-compat}
\lean{CategoryTheory.FilteredTriangulated.ForgetFiltration_iso_comp}
\uses{def-FF-leftAdjoint, def-FF-rightAdjoint, def-FF-i}

The isomorphism $\omega\circ i \iso {\ungras}_{\Df}$ given by the
counit of the adjunction of Definition~\ref{def-FF-leftAdjoint} is the 
inverse of the isomorphism ${\ungras}_{\Df}\iso\omega\circ i$ given by the
unit of the adjunction of Definition~\ref{def-FF-rightAdjoint}

\end{proposition}

The composition in the other direction, i.e. $i\circ\omega$, is isomorphic to $\sigma_{[0,0]}$.
(Obviously, this is not an arbitrary isomorphism.)

\begin{definition}
\label{def-i-FF}
\lean{CategoryTheory.FilteredTriangulated.Functor_forgetFiltration}
\uses{def-FF}

An isomorphism $i\circ\omega\iso\sigma_{[0,0]}$.

\end{definition}


\section{Definition A.1.4 and Proposition A.1.5}

\begin{definition}
% Definition A.1.4
\label{def-Gr}
\lean{CategoryTheory.FilteredTriangulated.Gr}
\uses{def-FF, def-truncGELE}

For every $n\in\Z$, we set $\Gra^n = \omega\circ\sigma_{{n,n}}$.

\end{definition}

The functors $\Gra^n$ are triangulated:

\begin{definition}
\label{def-Gr-commShift}
\lean{CategoryTheory.FilteredTriangulated.instCommShiftGrInt}
\uses{def-Gr}

For every $n\in\Z$, the functor $\Gra^n$ commutes with the triangulated shifts.

\end{definition}

\begin{proposition}
\label{def-Gr-tri}
\lean{CategoryTheory.FilteredTriangulated.instIsTriangulatedGr}
\uses{def-Gr-commShift}

For every $n\in\Z$, the functor $\Gra^n$ is triangulated.

\end{proposition}

\begin{proposition}
\label{prop-Gr_le}
\lean{CategoryTheory.FilteredTriangulated.Gr_bounded_above}
\leanok
\uses{def-Gr}
For every $X$ in $\DF$, there exists $n\in\Z$ such that $\Gra^m X$ is zero if 
$m>n$.

\end{proposition}

\begin{proposition}
\label{prop-Gr_ge}
\lean{CategoryTheory.FilteredTriangulated.Gr_bounded_below}
\leanok
\uses{def-Gr}
For every $X$ in $\DF$, there exists $n\in\Z$ such that $\Gra^m X$ is zero if 
$m<n$.

\end{proposition}


We now state Proposition A.1.5.


\begin{definition}
% Proposition A.1.5(i)
\label{def-Gr-commShift2}
\lean{CategoryTheory.FilteredTriangulated.Gr_commShift}
\uses{def-Gr}

For every $n\in\Z$, the functor $\Gra^n$ intertwines the second shift of $\DF$ and the identity functor of $\Df$.

\end{definition}

\begin{proposition}
% Proposition A.1.5(ii)
\label{prop-Gr-pure1}
\lean{CategoryTheory.FilteredTriangulated.Gr_pure_zero_of_ne_zero}
\uses{def-Gr}

For every $n\in\Z$ such that $n\ne 0$, for every $X$ in $\Df$, the object $\Gra^n(i(X))$ is zero. 

\end{proposition}

\begin{definition}
% Proposition A.1.5(i)
\label{def-Gr-pure2}
\lean{CategoryTheory.FilteredTriangulated.Gr_pure_of_zero}
\uses{def-Gr}

An isomorphism $\Gra^0\circ i\iso \ungras_{\Df}$.

\end{definition}

Proposition A.1.5(iii) actually should say that some particular morphisms are isomorphisms,
so we state it like that.

\begin{proposition}
% Proposition A.1.5(iii)
\label{prop-Gr-truncLE1}
\lean{CategoryTheory.FilteredTriangulated.Gr_truncLE_zero}
\uses{def-Gr, def-truncLE}

For all $r,n\in\Z$ such that $n < r$, the functor $\Gra^r\circ\sigma_{\le n}$ is zero.

\end{proposition}

\begin{proposition}
% Proposition A.1.5(iii)
\label{prop-Gr-truncLE2}
\lean{CategoryTheory.FilteredTriangulated.isIso_Gr_truncLEπ}
\uses{def-Gr, def-truncLEπ}

For all $r,n\in\Z$ such that $r \le n$, the morphism $\Gra^r\to\Gra^r\circ\sigma_{\le n}$
given by $truncLEπ$ is an isomorphism.

\end{proposition}

\begin{proposition}
% Proposition A.1.5(iii)
\label{prop-Gr-truncGE1}
\lean{CategoryTheory.FilteredTriangulated.Gr_truncGE_zero}
\uses{def-Gr, def-truncGE}

For all $r,n\in\Z$ such that $r < n$, the functor $\Gra^r\circ\sigma_{\ge n}$ is zero.

\end{proposition}

\begin{proposition}
% Proposition A.1.5(iii)
\label{prop-Gr-truncGE2}
\lean{CategoryTheory.FilteredTriangulated.isIso_Gr_truncLEπ}
\uses{def-Gr, def-truncLEπ}

For all $r,n\in\Z$ such that $n \le r$, the morphism $\Gra^r\circ\sigma_{\ge n}\to\Gra^r$
given by $truncGEι$ is an isomorphism.

\end{proposition}



\section{Compatibility with filtered triangulated functors (Proposition A.1.8)}

Let $\DF,\DF'$ be filtered triangulated categories, and $FT:\DF\to\DF'$ be a filtered 
triangulated functor.

Proposition A.1.8 is a mess as written.
It is not precise, the natural isomorphisms are not arbitrary!
Also, the square with $\sigma_{\ge n}$ is missing, and we need more squares
$\sigma_{[a,b]}$, as well as compatibilities with the connecting
morphisms in the triangles of $\sigma_{[a,b]}$.

By Propositions~\ref{prop-commute_truncLE_iso} and~\ref{prop-commute_truncGE_iso},
the commutative squares for $\sigma_{\le n}$ and
$\sigma_{\ge n}$ follow from the facts that a filtered triangulated functor is triangulated 
and preserves all the categories $\DF(\le m)$ and $\DF(\ge m)$.

The squares for the functors $\sigma_{[a,b]}$ and $\Gra^n$ then follow from the ones for
the truncation functors.

The square for $\omega$ is harder to define, as the definition (and characterization) 
of $\omega$ is a little complicated. There should be some compatibilities there, but I am 
not yet sure what they are. We will see what is needed later.

The "commutative square" for $\sigma_{\le n}$: 
\begin{tikzcd}
\DF\ar[r, "FT"]\ar[d, "\sigma_{\leq n}"'] &\DF'\ar[d, "\sigma_{\leq n}"]
\\
\DF\ar[r, "FT"'] &\DF'
\end{tikzcd}

\begin{proposition}
% Proposition A.1.8
\label{prop-FT-truncLE}
\lean{CategoryTheory.FilteredTriangulated.liftFunctor_truncLE_comm}
\uses{def-commute_truncLE, def-FilteredTriangulatedFunctor}

Let $n\in\Z$. The morphism $\sigma_{\le n}\circ FT \to FT\circ\sigma_{\le n}$
of Definition~\ref{def-commute_truncLE} is an isomorphism.

\end{proposition}

The "commutative square" for $\sigma_{\ge n}$: 
\begin{tikzcd}
\DF\ar[r, "FT"]\ar[d, "\sigma_{\geq n}"'] &\DF'\ar[d, "\sigma_{\geq n}"]
\\
\DF\ar[r, "FT"'] &\DF'
\end{tikzcd}

\begin{proposition}
% Proposition A.1.8
\label{prop-FT-truncGE}
\lean{CategoryTheory.FilteredTriangulated.liftFunctor_truncGE_comm}
\uses{def-commute_truncGE, def-FilteredTriangulatedFunctor}

Let $n\in\Z$. The morphism $FT\circ\sigma_{\ge n}\to\sigma_{\ge n}\circ FT$
of Definition~\ref{def-commute_truncGE} is an isomorphism.

\end{proposition}

\begin{definition}
% Proposition A.1.8
\label{def-FT-Gr}
\lean{CategoryTheory.FilteredTriangulated.lifting_Gr_comm}
\uses{def-Gr, def-FilteredTriangulatedFunctor}

For $n\in\Z$, an isomorphism $\Gra^n\circ FT\to FT\circ\Gra^n$.

\end{definition}
\begin{tikzcd}
\DF\ar[r, "FT"]\ar[d, "\Gra^n"'] &\DF'\ar[d, "\Gra^n"] &&
\\
\Df\ar[r, "T"'] &\Df'&&
\end{tikzcd}

\begin{definition}
% Proposition A.1.8
\label{def-FT-FF}
\lean{CategoryTheory.FilteredTriangulated.lifting_forgetFiltrating_comm }
\uses{def-FF, def-FilteredTriangulatedFunctor}

An isomorphism $\omega\circ FT\to FT\circ\omega'$.

\end{definition}

\begin{tikzcd}
\DF\ar[r, "FT"]\ar[d, "\omega"'] &\DF'\ar[d, "\omega"] &&
\\
\Df\ar[r, "T"'] & \Df'&&
\end{tikzcd}



