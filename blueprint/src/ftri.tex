\section{Section A.1: filtered triangulated categories}

In Appendix~A of~\cite{Be1}, Beilinson introduced f-categories over triangulated categories, that have all the abstract properties of filtered
derived categories, and generalized the properties of filtered derived categories to this more general setting. We review his definition and results.
Note that Section~6 of Schn\"urer's paper \cite{Schnur} gives more detailed proofs of many of the results of Appendix~A of~\cite{Be1}.

When formalizing the definition of f-categories, we run into our first issue immediately. Beilinson defines an f-category as a triangulated
category $C$ with a second shift functor $s$, which should be a triangulated self-equivalence (plus some extra structure). In particular,
the functor $s$ needs to commute with the shifts coming from the triangulated category structure. The easiest way to encode all
the necessary compatibilities is actually to put a shift by $\Z \times \Z$ on our category $C$, where the shift by the first factor
will be part of the triangulated structure and the shift by the second factor will give the functor $s$. The beginning of the project
is devoted to setting this shift structure. In particular, we chose to make the default shift by $\Z$ on the  category $C$ to be the
shift by the first factor, and to introduce a type synonym `FilteredShift C' which carries a shift by $\Z$ encoding the functor  $s$.

The following is Definition~A.1 of~\cite{Be1}.

\begin{definition}
\label{def-FilteredTriangulated}
\lean{CategoryTheory.FilteredTriangulated}
We introduce the following objects:
\begin{itemize}
\item[(1)] A \textit{filtered} \textit{triangulated} \textit{category}, or for short a \emph{f-category}, is the data of:
\begin{itemize}
\item a triangulated category $\DF$;
\item two full triangulated subcategories $\DF(\leq 0)$, $\DF(\geq 0)$ of $\DF$ that are stable by isomorphisms;
\item a triangulated self-equivalence $s\colon \DF\ra\DF$ (called \emph{shift of filtration});
\item a morphism of functors $\alpha\colon \id_{\DF}\ra s$;

\end{itemize}
satisfying the following conditions, where, for every $n\in\Z$, we set
\[\DF(\leq n)=s^n\DF(\leq 0)\mbox{ and }\DF(\geq n)=s^n\DF(\geq 0):\]
\begin{itemize}
\item[(i)] We have $\DF(\geq 1)\subset\DF(\geq 0)$, $\DF(\leq 1)\supset\DF(\leq 0)$ and
\[\DF=\bigcup_{n\in\Z}\DF(\leq n)=\bigcup_{n\in\Z}\DF(\geq n).\]
\item[(ii)] For any $X\in\Ob\DF$, we have $\alpha_X=s(\alpha_{s^{-1}(X)})$.
\item[(iii)] For any $X\in\Ob\DF(\geq 1)$ and $Y\in\Ob\DF(\leq 0)$, we have $\Hom(X,Y)=0$, and the maps $\Hom(s(Y),X)\ra\Hom(Y,X)\ra\Hom(Y,s^{-1}(X))$ induced
by $\alpha_Y$ and $\alpha_{s^{-1}(X)}$ are bijective.
\item[(iv)] For every $X\in\Ob\DF$, there exists a distinguished triangle $A\ra X\ra B\flnom{+1}$ with $A$ in $\DF(\geq 1)$ and $B$ in $\DF(\leq 0)$.
\end{itemize}

\item[(2)] If $\DF$ and $\DF'$ are f-categories, an \emph{f-functor} from $\DF$ to $\DF'$ is the data of a triangulated functor $T\colon \DF\ra\DF'$ and a natural isomorphism
$s'\circ T\iso T\circ s$ 
such that $T(\DF(\leq 0))\subset\DF'(\leq 0)$,
$T(\DF(\geq 0))\subset\DF(\geq 0)$ and that, for every $X\in\Ob\DF$, the following triangle commutes:

\begin{tikzcd}
T(X) \arrow[r, "\alpha'_{T(X)}"] \arrow[rd, "T(\alpha(X))"'] 
& s'(T(X))\arrow[d, "\wr"] \\
& T(s(X))
\end{tikzcd}

\item[(3)] 
Let $\Df$ be a triangulated category. An \emph{f-category over $\Df$} is an f-category $\DF$ together with an equivalence $i\colon \Df\ra\DF(\leq 0)\cap\DF(\geq 0)$.
If $\Df'$ is another triangulated category, $\DF'$ is an $f$-category over $\Df'$ and $T\colon \Df\ra\Df'$ is a triangulated functor, an \emph{f-lifting} of $T$ is an f-functor
$FT\colon \DF\ra\DF'$ and a natural isomorphism $i'\circ T\simeq TF\circ i$.

\end{itemize}
\label{def_f_category}
\end{definition}


\begin{proposition}[Proposition~A.3 of~\cite{Be1}]
Let $\DF$ be an f-category.
\begin{itemize}
\item[(i)] For every $n\in\Z$, the inclusion $\DF(\leq n)\subset\DF$ admits a left adjoint $\sigma_{\leq n}$, and the inclusion $\DF(\geq n)\subset\DF$ admits a right adjoint $\sigma_{\geq n}$.
The functors $\sigma_{\leq n}$, $\sigma_{\geq n}$ are triangulated and preserve the subcategories $\DF(\leq m)$, $\DF(\geq m)$ for every $m\in\Z$.
\item[(ii)] For $a,b\in\Z$, there exists a unique isomorphism of functors $\sigma_{\leq a}\sigma_{\geq b}\simeq\sigma_{\geq b}\sigma_{\leq a}$ that makes the following diagram commute:

\begin{tikzcd}
\sigma_{\geq b}\ar[rr]\ar[rd] & & \id_{\DF}\ar[rr] & & \sigma_{\leq a} \\
& \sigma_{\leq a}\sigma_{\geq b}\ar[rr, "\sim"'] && \sigma_{\geq b}\sigma_{\leq a}\ar[ru] &
\end{tikzcd}

\item[(iii)] Let $X\in\Ob\DF$. Then there exists a unique morphism $\delta\colon \sigma_{\leq 0}X\ra\sigma_{\geq 1}[1]$ making the triangle $\sigma_{\leq 1}X\ra X\ra\sigma_{\leq 0}X\flnom{\delta}\sigma_{\geq 1}X[1]$
distinguished.
Any distinguished triangle $A\ra X\ra B\flnom{+1}$ with $A\in\Ob\DF(\geq 1)$ and $B\in\Ob\DF(\leq 0)$ admits a unique isomorphism to the triangle of the previous sentence.

\item[(iv)] We have canonical isomorphisms $\sigma_{\leq n}\circ s=s\circ\sigma_{\leq n-1}$ and $\sigma_{\geq n}\circ s=s\circ\sigma_{\geq n-1}$.

\end{itemize}

\label{prop_f_category1}
\end{proposition}

Point (iv) is not stated in Proposition~A.3 of~\cite{Be1} but follows immediately from the fact that $s(\DF(\leq n -1))=\DF(\leq n)$ (resp. $s(D(\geq n-1))=D(\geq n)$) and the uniqueness
of adjoints.

\begin{definition}
Let $\Df$ be a triangulated category and $\DF$ be an f-category over $\Df$. For every $n\in\Z$, we define a functor $\Gra^n\colon \DF\ra\Df$ by
$\Gra^n=i^{-1}\circ s^{-n}\circ\sigma_{\leq n}\sigma_{\geq n}$.

\end{definition}

\begin{proposition}
Let $\Df$ be a triangulated category and $\DF$ be an f-category over $\Df$. 
\begin{itemize}
\item[(i)] For every $r\in\Z$, we have a natural isomorphism $\Gra^r\circ s=\Gra^{r-1}$. 

\item[(ii)] Let $r\in\Z$. Then $\Gra^r\circ i=0$ if $r\ne 0$ and $\Gra^r\circ i\simeq\id_{\Df}$ if $r=0$.

\item[(iii)] Let $r,n\in\Z$. We have
\[\Gra^r\circ\sigma_{\leq n}=\left\{\begin{array}{ll}\Gra^r& \mbox{if }r\leq  n\\ 0 & \mbox{otherwise}\end{array}\right.
\qquad\mbox{and}\quad\Gra^r\circ\sigma_{\geq n}=\left\{\begin{array}{ll}\Gra^r& \mbox{if }r\geq  n\\ 0 & \mbox{otherwise}\end{array}\right..\]

\end{itemize}
\label{prop_Gr}
\end{proposition}

\begin{proof}
Point (i) follows from Proposition~\ref{prop_f_category1}(iv), point (ii) from the fact that the image of $i$ is contained in $\DF(\leq 0)\cap\DF(\geq 0)$, and point (iii)
from the definition of $\Gra^r$.
\end{proof}

\begin{proposition}[Proposition~A.3 of~\cite{Be1}]
Let $\Df$ be a triangulated category and $\DF$ be an f-category over $\Df$.
Then there exists a triangulated functor $\omega\colon \DF\ra\Df$ such that:\footnote{Note that there is a typo in Proposition~A.3 of~\cite{Be1}: the left and right
adjoints are switched; see the correction in Proposition~6.6 of~\cite{Schnur}.}
\begin{itemize}
\item[(a)] $\omega_{\mid\DF(\leq 0)}\colon \DF(\leq 0)\ra\Df$ is left adjoint to $\Df\flnom{i}\DF(\leq 0)\cap\DF(\geq 0)\subset\DF(\leq 0)$;
\item[(b)] $\omega_{\mid\DF(\geq 0)}\colon \DF(\geq 0)\ra\Df$ is right adjoint to $\Df\flnom{i}\DF(\leq 0)\cap\DF(\geq 0)\subset\DF(\geq 0)$;
\item[(c)] for any $X\in\Ob\DF$, the map $\omega(\alpha_X)\colon \omega(X)\ra\omega(s(X))$ is an isomorphism;
\item[(d)] if $A\in\Ob\DF(\leq 0)$ and $B\in\DF(\geq 0)$, then $\omega\colon \Hom(A,B)\ra\Hom(\omega(A),\omega(B))$ is bijective.

\end{itemize}
Moreover, $\omega$ is determined up to unique isomorphism by properties (a) and (c) (resp. (b) and (c)).

\label{prop_f_category2}
\end{proposition}

The following proposition follows easily from the definitions.

\begin{proposition}
Let $\Df,\Df'$ be triangulated categories, and let $\DF$ (resp. $\DF'$) be an f-category over $\Df$ (resp. $\Df'$).
Let $T\colon \Df\ra\Df'$ be a triangulated functor, and let $FT\colon \DF\ra\DF'$ be an f-lifting of $T$.
Then the following squares commute up to natural isomorphism:

\begin{tikzcd}
\DF\ar[r, "FT"]\ar[d, "\omega"'] &\DF'\ar[d, "\omega"] &&
\DF\ar[r, "FT"]\ar[d, "\Gra^n"'] &\DF'\ar[d, "\Gra^n"] &&
\DF\ar[r, "FT"]\ar[d, "\sigma_{\leq n}"'] &\DF'\ar[d, "\sigma_{\leq n}"]
\\
\Df\ar[r, "T"'] & \Df'&&
\Df\ar[r, "T"'] &\Df'&&
\DF\ar[r, "FT"'] &\DF'
\end{tikzcd}

\label{prop_f_lifting}
\end{proposition}


%\begin{subprop}
%Let $\Df$ be a triangulated category, $\DF$ be an f-category over $\Df$, and $\Df'$ be a full triangulated subcategory of $\Df$ that is stable by isomorphisms. We define a full
%subcategory $\DF'$ of $\DF$ by
%\[\Ob\DF'=\{K\in\Ob\Df\mid\forall r\in\Z,\ \Gra^rK\in\Ob\Df'\}.\]
%Then $\DF'$ is a triangulated subcategory of $\DF$, it is stable by isomorphisms, we have $s(\DF')\subset\DF'$ and $i(\Df')\subset\DF'$. The data of
%$\DF'$, $\DF'\cap\DF(\leq 0)$, $\DF'\cap\DF(\geq 0)$, $s_{\mid\DF'}\colon \DF'\ra\DF'$, $\alpha$ and $i_{\mid\Df'}\colon \Df'\ra\DF'$ defines an f-category over $\Df'$.

%\label{prop_f_category3}
%\end{subprop}

%\begin{proof}
%As the functors $\Gra^r$ are triangulated, $\DF'$ is a triangulated subcategory of $\DF$; it is clearly stable by isomorphisms, and it stable by $s$ thanks to the isomorphisms
%$\Gra^r\circ s=\Gra^{r-1}$ (Proposition~\ref{prop_Gr}(i)). 
%If $X\in\Ob\Df'$, then $\Gra^r(i(X))=0$ if $r\ne 0$ and $\Gra^0(i(X))\simeq X$ (Proposition~\ref{prop_Gr}(ii)), so $i(X)\in\Ob\DF'$. To prove the last assertion, we check the conditions of
%Definition~\ref{def_f_category}. Conditions (i)-(iii) are clear. To check condition~(iv), it suffices by Proposition~\ref{prop_f_category1}(iii) to prove that the functors
%$\sigma_{\leq n}$, $\sigma_{\geq n}$ preserve $\DF'$; but this follows immediately from Proposition~\ref{prop_Gr}(iii).
%\end{proof}

